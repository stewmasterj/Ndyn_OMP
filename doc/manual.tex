\documentclass[a4paper,11pt]{scrbook}              % Book class in 11 points
\usepackage[english]{babel}
\parindent0pt  \parskip10pt             % make block paragraphs

% Note that book class by default is formatted to be printed back-to-back.
%%%%%%%%%%%%%%%%%%%%%%%%%%%%%%%%%%%%%%%%%%%%%%%%%%%%%%%%%%%%%%%%%%%%%%%%%%%%%%%%
\begin{document}                        % End of preamble, start of text.
\frontmatter                            % only in book class (roman page #s)
%\titlehead{Some kind of text like a location or something}
\subject{Manual}
\title{Newtonian Dynamics -- Ndyn}    % Supply information
\subtitle{Multithreaded enabled via OpenMP}
\author{Ross J. Stewart}              %   for the title page.
\date{\today}                           %   Use current date. 
%\publishers{publishers} 
\lowertitleback{This book was set with the help of {\KOMAScript} and {\LaTeX}}
\dedication{Dedicated to those who wish to learn how to use this program.}
\maketitle                              % Print title page.
\tableofcontents                 

\chapter*{Abstract} %{{{
\addcontentsline{toc}{chapter}{Abstract}
This is a molecular dynamic program that is multithreaded for use on shared memory
 machines. This does not use domain decomposition like most MPI MD codes do.
The various threads simply share the primary particle loops.
This code only utilizes particle pair potentials, no multibody potentials have
 been enabled.
%}}}

\mainmatter                             % only in book class (arabic page #s)
%%%%%%%%%%%%%%%%%%%%%%%%%%%%%%%%%%%%%%%%%%%%%%%%%%%%%%%%%%%%%%%%%%%%%%%%%%%%%%%%
%\part{Input File Commands} %==

\chapter{Input File Commands}
Manditory arguments in \verb|[]|, optional ones in \verb|{}|.
If a directive was typed incorrectly, just type it again.

\begin{description}
\item [help] display the help message
\item [Model] \verb|[FILE]|       Specify the model file with the particle position data. 
                    if not specified, will place random particles in box.
\item [Seed] \verb|[i]|           seed for random number generator
\item [NTotal] \verb|[i]|         Total number of particles, if no Model file.
\item [BoxSize] \verb|[r][r]{r}|  Total size of simulation box in each direction
\item [Velocity] \verb|[r] {i}|   set initial velocity distribution to target kinetic energy
                    density [r] KE/N for group {i}, set this to 3/2*k*T,
                    convert the output KE to T=2*KE/(3*N*k)
\item [ReNeighbor] \verb|[b] {i}|  Reneighbor, ([b]=T/F), DEFAULT=T, check frequency {i}=1
\item [NeiMax] \verb|[i]|         Maximum number of neighbours, default=90.
\item [Types] \verb|[i]|          Number of particle types. \\
\verb| [c4] [r] [r]|      List of each particle type Name, Mass and charge
\item [Pairs] \verb|[i] {r} {i}|  Number of pair potential interactions. rMin, Table size. \\
\verb| [c4] [c4] [sym|asym] [c4] {rN}|   List of interactions between particle
                    types, if they're [i-j]=[j-i] or not, potential type,
                    list of N potential parameters, dependent on the type.
\item [Isokinetic] \verb|[r] [r] {i}| Set a constant kinetic energy density KE/N [r] with
                     parm [r] every [i] steps
\item [Isoenergetic] \verb|[r] {i}| Maintian initial total energy with parm [r] every [i] steps
\item [Isobaric] \verb|[r(3)] [r] {i}| Set a constant pressure vector [r(3)] with parm [r] every [i] steps
\item [Anisokinetic]       Unset the kinetistat
\item [Anisoenergetic]     Unset the energetistat
\item [Anisobaric]         Unset the barostat
\item [Group] \verb|[c8] [type|dom]| parameters   specify a set of atoms as in one group
                    either by atom type or geometric domain
\item [Hold] \verb|[c8] [type|dom]| parameters   specify a set of atoms as in one HoldGroup
                    either by atom type or geometric domain
\item [ChangeType] \verb|[c8] [c4]| Set the atoms in group [c8] to the atom type [c4]
\item [Disp] \verb|[c8] [r(3)|l(3)]| Apply a displacement vector to a HoldGroup [c8]
\item [BoxVel] \verb|[r(3)]|      Change box dimensions at a constant rate
\item [StopBoxVel]         stop changing the box by unsetting BoxVel
\item [ConfOutFreq] \verb|[i]|    Number of timesteps between configuration outputs
\item [StepNumber] \verb|[i]|     Step number to begin a run with, useful when restarting
\item [Run] \verb|[i] {r}|        Run for [i] steps with [r] deltat
\item [Exec]  \verb|[s]|          Executes [s] as a system command
\item [END] or {\bf{DONE}}        Marks the end of the directive list, then begins run.
\item [EXIT] or {\bf{exit}} or {\bf{quit}} or {\bf{abort}}    Will Terminate the program

\end{description}

\chapter{Pair Potentials}
Pair potential types and parameters, last parameter is cutoff
\begin{description}
\item [sine]               rcut=PI/2 [Default]
\item [soft] rc A          soft cosine useful for overlapping atoms
\item [spring] rc D0 sc Vol  spring potential using strain based on initial configuration
\item [morse] rc D0 alpha r0  Morse pair potential
\item [buck]  rc A C rho   Buckingham pair potential
\end{description}

\chapter{Preferred Units}
\begin{tabular}{ l l }
length:       &  1 Angstrom \\
energy:       &  1 electron volt [eV] \\
mass:         &  1 atomic mass unit [amu] \\
time:         &  10.1805057e-15 (sec) 10.1805057 [fs] \\
temperature:  &  1 Kelvin [K] \\
pressure:     &  1 eV*Ang\^-3 = 160.217662080e9 Pa \\
Boltzmann Constant: & k=8.6173303e-5 eV/K \\
charge:       &  1 e*3.794685 = e*sqrt(14.399637 Ang*eV/e\^2) : [e*sqrt(Ang*eV)] \\
\end{tabular}

To convert total $E_K$ to $T$ 
\begin{equation}
   T = \frac{2E_K}{3Nk}
\end{equation}
\begin{equation}
   \frac{E_K}{N} = \frac{3}{2}Tk 
\end{equation}

\chapter{Features Implemented}
 Implement "Newton's third law", since it's SMP. \\
 PBCs make sure the distance routine works with PBCs. \\
 Normalize positions to box lengths \\
 VNLmod or CLLs? \\
 various potential types, \\
   including charges on atoms \\
 Update list mechanism \\
 initial velocities \\
 temperature control (rescale and berendsen) \\
 pressure control (berendsen, nose-hoover?) \\
 user defined groups based either on atom type or geometric domain \\
 apply boundary condition types to defined groups \\
  initial kinetic energy density (Velocity KEdens [GROUPNAME]) \\
  follow neighbours, use asymmetric spring potential \\
  change group type (ChangeType GROUPNAME NEWTYPENAME) \\
  fixed(disp=0) \\
To be implemented
o  vel (or just fix and use BoxVel) \\
o  force \\
o Thermostat group \\
o Langevin damping 

\chapter{Tips and Caveats}
\begin{enumerate}
\item Only symmetric pair interactioins are counted in system energy and virial totals
\item Only pairs that interact via a pair potential are included in neighbour lists
\item there are different types of particle sets that can be defined
  \subitem 'Hold' domains are used to apply boundary conditions and loads during dynamics
  \subitem 'Groups' are genrally temporary and used to set initial attributes, like velocities
     or maybe local ensembles? like temperature control with body force?
  \subitem ?? reneighbouring can be performed on selected atom types during dynamics.
\end{enumerate}



%%%%%%%%%%%%%%%%%%%%%%%%%%%%%%%%%%%%%%%%%%%%%%%%%%%%%%%%%%%%%%%%%%%%%%%%%%%%%%%%
\end{document}                          % The required last line
%%%%%%%%%%%%%%%%%%%%%%%%%%%%%%%%%%%%%%%%%%%%%%%%%%%%%%%%%%%%%%%%%%%%%%%%%%%%%%%%
